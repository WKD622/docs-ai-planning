\documentclass{article}
\usepackage[a4paper, margin=1in]{geometry}
\usepackage{setspace}
\setstretch{1.25}


\begin{document}


\section{Exercise 1}

Jakub Ziarko, Muhammad Irfan, Muhammad Kashan

\subsection{Computing task}

Grid is $10 x 10$, that means we have $100$ fields.
\newline
$2$ - for these many times we can choose status for the field (dirty or clean). We have 100 fields, so
we have to apply this rule to 100 fields:
\newline
$2^{100}$ - this is number of all combinations of board's settings.
\newline
$100^{5}$ - we have 5 robots. Each of them we can put on a board in 100 ways. 
\newline
$20^5$ - each robot has 20 charge levels.
\newline 
Number of all states is a multiplication of values mentioned above: 
$S=2^{100}\cdot100^{5}\cdot20^5$
\newline
$T=S \cdot 10^{-6}s= 2^{100} \cdot 100^{5} \cdot 20^5 \cdot 10^{-6}s = 2^{100} \cdot 10^4 \cdot 20^5s$ 



\subsection{Questions}

\begin{enumerate}
	\item { In the start it says that Planning works when "Your problem is subject to
	 frequent change", but heuristics by its nature enforce a rigidity in description of the problem. 
	 Is this an intended trade-off or if it is an open research problem? }
	\item { What was so special about LM-cut heuristic that it coused a major boost in optimal
	 planning in the last years?}
\end{enumerate}
\end{document} 